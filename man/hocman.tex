
\magnification\magstep1
\input fixpdfmag % for pdftex

\hsize 15 true cm \advance\hoffset 5 true mm
\vsize 24 true cm \advance\voffset 1 true mm

\parskip=1pt plus 1pt
\parindent=1pc

\font\tfone=cmr17
\font\tftwo=cmr12
\font\tenrm=cmr10
\font\eightrm=cmr8
\font\eighttt=cmtt8

\def\hoc{{\tenrm H\eightrm OC}}%
\catcode`\|=\active\let|=\vert% alternatives in grammars
\def\section #1\par{\vskip 0pt plus .1\vsize\penalty-250
  \vskip 0pt plus -.1\vsize \vskip 18pt plus 6pt minus 4pt
  \vskip\parskip \leftline{\bf #1}%
  \nobreak\medskip\noindent\ignorespaces}%
\def\boxit#1{\vbox{\hrule\hbox{\vrule\kern6pt
  \vbox{\kern6pt#1\kern3pt}\kern3pt\vrule}\hrule}}%
\def\0{\phantom{0}}%
\def\\{\char"5C}% backslash in tt font

\def\begincode{\begingroup\tt\obeylines\obeyspaces\frenchspacing
  \catcode`\$=12\catcode`\{=12\catcode`\}=12\catcode`\^=12\relax}%
\def\endcode{\par\endgroup}


\topglue 0pc\noindent
This is {\bf Appendix~2: Hoc Manual} in
{\sl The UNIX Programming Environment}.
\vskip 2pc plus 1pc

\centerline{\tfone H\tftwo OC}
\medskip
\centerline{\tftwo An Interactive Language For Floating Point Arithmetic}
\bigskip
\centerline{\it Brian Kernighan}
\centerline{\it Rob Pike}

\vskip 2pc plus 6pt minus 4pt

\begingroup\narrower\narrower
\noindent{\bf Abstract.\enspace}\ignorespaces
Hoc is a simple programmable interpreter for floating point expressions.
It has C-style control flow, function definition and the usual
numerical built-in functions such as cosine and logarithm.\par
\endgroup

\section 1. Expressions

\hoc\ is an expression language, much like~C: although there are several
control-flow statements, most statements such as assignments are
expressions whose value is disregarded. For example, the assignment
operator {\tt=} assigns the value of its right operand to its left
operand, and yields the value, so multiple assignments work.
The expression grammar is:
$$
\vbox{\halign{\tabskip1em\it#\hfil&$#$\hfil&\it#\hfil\cr
  expr:& &number\cr
      &|&variable\cr
      &|&{\tt(} expr {\tt)}\cr
      &|&expr binop expr\cr
      &|&unop expr\cr
      &|&function {\tt(} arguments {\tt)}\cr}}
$$
Numbers are floating point. The input format is that recognized
by {\tt scanf}(3): digits, decimal point, digits, {\tt e} or {\tt E},
signed exponent. At least one digit or a decimal point must be
present; the other components are optional.

Variable names are formed from a letter followed by a string of
letters and numbers. {\it binop\/} refers to binary operators such
as addition or logical comparison; {\it unop\/} refers to the two
negation operators, {\tt!} (logical negation, `not') and {\tt-}
(arithmetic negation, sign change). Table~1 lists the operators.
$$
\boxit{\vbox{\halign{\tabskip1em\tt\hfil#\hfil&#\hfil\cr
  \omit\span\hfil{\bf Table~1:} Operators, in decreasing order of precedence\cr
  \noalign{\smallskip}
  \char"5E&  exponentiation, right associative\cr
  !  -&      (unary) logical and arithmetic negation\cr
  * \ /&     multiplication, division\cr
  + \ -&     addition, subtraction\cr
  > \ >=&    relational operators: greater, greater or equal,\cr
  < \ <=&    \quad less, less or equal,\cr
  == !=&     \quad equal, not equal (all same precedence)\cr
  \&\&&      logical {\eightrm AND} (both operands always evaluated)\cr
  \char"7C\char"7C & logical {\eightrm OR} (both operands always evaluated)\cr
  =&         assignment, right associative\cr}}}
$$

Functions, as described later, may be defined by the user.
Function arguments are expressions separated by commas.
There are also a number of built-in functions, all of which
take a single argument, described in Table~2.
$$
\boxit{\vbox{\halign{\tabskip1em\tt#\hfil&#\hfil\cr
  \omit\span\hfil{\bf Table~2:} Built-in Functions\cr
  \noalign{\smallskip}
  abs($x$)&    $|x|$, absolute value of $x$\cr
  atan($x$)&   arc tangent (in radians) of $x$\cr
  cos($x$)&    $\cos x$, cosine of $x$, $x$ in radians\cr
  exp($x$)&    $e^x$, exponential of $x$\cr
  int($x$)&    integer part of $x$, truncated towards zero\cr
  log($x$)&    $\log x$, logarithm base $e$ of $x$\cr
  log10($x$)&  $\log_{10}x$, logarithm base 10 of $x$\cr
  sin($x$)&    $\sin x$, sine of $x$, $x$ in radians\cr
  sqrt($x$)&   $\sqrt x$, $x^{1/2}$, square root of $x$\cr}}}
$$

Logical expressions have value $1.0$ (true) and $0.0$ (false).
As in~C, any non-zero value is taken to be true. As is always
the case with floating point numbers, equality comparisons are
inherently suspect.

\hoc\ also has a few built-in constants, shown in Table~3.
$$
\boxit{\vbox{\halign{\tabskip1em\tt#\hfil&#\hfil&#\hfil\cr
  \omit\span\span\hfil{\bf Table~3:} Built-in Constants\hfil\cr
  \noalign{\smallskip}
  DEG&     57.29577951308232087680& $180/\pi$, degrees per radian\cr
  E&      \02.71828182845904523536& $e$, base of natural logarithms\cr
  GAMMA&  \00.57721566490153286060& $\gamma$, Euler-Mascheroni constant\cr
  PHI&    \01.61803398874989484820& $(\sqrt 5+1)/2$, the golden ratio\cr
  PI&     \03.14159265358979323846& $\pi$, circular transcendental number\cr}}}
$$

\section 2. Statements and Control Flow

\hoc\ statements have the following grammer:
$$
\vbox{\halign{\tabskip1em\it#\hfil&$#$\hfil&\it#\hfil\cr
  stmt:& &expr\cr
      &|&variable {\tt=} expr\cr
      &|&procedure {\tt(} arglist {\tt)}\cr
      &|&while {\tt(} expr {\tt)} stmt\cr
      &|&if {\tt(} expr {\tt)} stmt\cr
      &|&if {\tt(} expr {\tt)} stmt {\tt else} stmt\cr
      &|&{\tt\char"7B} stmtlist {\tt\char"7D}\cr
      &|&{\tt print} expr-list\cr
      &|&{\tt return} optional-expr\cr
  \noalign{\medskip}
  stmtlist:& & {\rm (nothing)}\cr
      &|& stmtlist stmt\cr}}
$$
An assignment is parsed by default as a statement rather than
as an expression, so assignments typed interactively do not print
their value.

Note that semicolons are not special to \hoc: statements are
terminated by newlines. This causes some peculiar behavior.
The following are legal {\tt if\/} statements:

\medskip
\begincode
if (x < 0) print(y) else print (z)
\medskip
if (x < 0) {
~      print(y)
} else {
~      print(z)
}
\endcode
\medskip

\noindent
In the second example, the braces are mandatory: the newline
after the {\tt if} would terminate the statement and produce
a syntax error were the brace omitted.

The syntax and semantics of \hoc\ control flow facilities are
basically the same as in~C. The {\tt while} and {\tt if}
statements are just as in~C, except there are no {\tt break}
or {\tt continue} statements.

\section 3. Input and Output: {\tt read} and {\tt print}

The input function {\tt read}, like the other built-ins, takes
a single argument. Unlike the built-ins, though, the argument
is not an expression: it is the name of a variable. The next
number (as defined above) is read from the standard input and
assigned to the named variable. The return of {\tt read} is
1 (true) if a value was read, and 0 (false) if {\tt read}
encountered end of file or an error.

Output is generated with the {\tt print} statement.
The arguments to {\tt print} are a comma-separated list
of expressions and strings in double quotes, as in~C.
Newlines must be supplied; they are never provided
automatically by {\tt print}.

Note that {\tt read} is a special built-in function, and
therefore takes a single parenthesized argument, while
{\tt print} is a statement that takes a comma-separated,
unparenthesized list:

\medskip
\begincode
while (read(x)) {
~       print "value is ", x, "\\n"
}
\endcode

\section 4. Functions and Procedures

Functions and procedures are distinct in \hoc, although they
are defined by the same mechanism. This distinction is simply
for run-time error checking: it is an error for a procedure
to return a value, and for a function {\it not\/} to return one.

The definition syntax is:
$$
\vbox{\halign{\tabskip1em\llap{#}&\quad\it#\hfil\cr
 function:&  {\tt func} name {\tt()} stmt\cr
 procedure:& {\tt proc} name {\tt()} stmt\cr}}
$$
{\it name\/} may be the name of any variable---built-in functions
are excluded. The definition, up to the opening brace or statement,
must be on one line, as with the {\tt if} statement above.

Unlike~C, the body of a function or procedure may be any statement,
not necessarily a compound (brace-enclosed) statement. Since
semicolons have no meaning in \hoc, a null procedure body is formed
by an empty pair of braces.

Functions and procedures may take arguments, separated by commas,
when invoked. Arguments are referred to as in the shell: {\tt\$3}
refers to the third (1-indexed) argument. They are passed by value
and within functions are semantically equivalent to variables.
It is an error to refer to an argument numbered greater than the
number of arguments passed to the routine. The error checking is
done dynamically, however, so a routine may have variable numbers
of arguments if initial arguments affect the number of arguments
to be referenced (as in C's {\tt printf}).

Functions and procedures may recurse, but the stack has limited
depth (about a hundred calls). The following shows a \hoc\ definition
of Ackermann's function:

\medskip
\begincode
$ hoc
func ack() {
~       if ($1 == 0) return $2+1
~       if ($2 == 0) return ack($1-1, 1)
~       return ack($1-1, ack($1, $2-1))
}
ack(3, 2)
~       29
ack(3, 3)
~       61
ack(3, 4)
hoc: stack too deep near line 8
\endcode

\section 5. Examples

Stirling's formula:\quad\smash{$\displaystyle
n!\sim\sqrt{2n\pi}\left(n\over e\right)^n\left(1+{1\over12n}\right)$}

\medskip
\begincode
$ hoc
func stirl() {
~   return sqrt(2*$1*PI) * ($1/E)^$1 * (1 + 1/(12*$1))
}
stirl(10)
~       3628684.7
stirl(20)
~       2.4328818e+18
\endcode
\medskip
\noindent
Ratio of factorial to Stirling approximation:
\medskip
\begincode
func fac() if ($1 <= 0) return 1 else return $1 * fac($1-1)
i = 9
while ((i = i+1) <= 20) {
~       print i, "  ", fac(i)/stirl(i), "\\n"
}
\endcode
\medskip
\line{(Expected output to the right.)\hfil
\smash{\vbox{\eighttt\baselineskip 9pt
\halign{\tabskip1em#\hfil&#\hfil\cr
10& 1.0000318\cr
11& 1.0000265\cr
12& 1.0000224\cr
13& 1.0000192\cr
\noalign{\vskip20pt}
14& 1.0000166\cr
15& 1.0000146\cr
16& 1.0000128\cr
17& 1.0000114\cr
18& 1.0000102\cr
19& 1.0000092\cr
20& 1.0000083\cr}}}}

\bye

